% This is samplepaper.tex, a sample chapter demonstrating the
% LLNCS macro package for Springer Computer Science proceedings;
% Version 2.20 of 2017/10/04
%
\documentclass[runningheads]{llncs}
%
\usepackage{amsmath}
\usepackage{booktabs} % For pretty tables
\usepackage{caption} % For caption spacing
\usepackage{subcaption} % For sub-figures
\usepackage{graphicx}
\usepackage{rotating}
\usepackage{pgfplots}
\usepackage[all]{nowidow}
\usepackage[utf8]{inputenc}
\usepackage[margin=1in]{geometry}
\usepackage{tikz}
\usetikzlibrary{er,positioning,bayesnet}
\usepackage{multicol}
\usepackage{algpseudocode,algorithm,algorithmicx}
\usepackage{minted}
\usepackage{hyperref}
\usepackage{siunitx}
\usepackage{esdiff}
\usepackage{float}
\usepackage[inline]{enumitem} % Horizontal lists
% Used for displaying a sample figure. If possible, figure files should
% be included in EPS format.
%
% If you use the hyperref package, please uncomment the following line
% to display URLs in blue roman font according to Springer's eBook style:
% \renewcommand\UrlFont{\color{blue}\rmfamily}

\newcommand{\card}[1]{\left\vert{#1}\right\vert}
\newcommand*\Let[2]{\State #1 $\gets$ #2}
\definecolor{blue}{HTML}{1F77B4}
\definecolor{orange}{HTML}{FF7F0E}
\definecolor{green}{HTML}{2CA02C}

\pgfplotsset{compat=1.14}

\renewcommand{\topfraction}{0.85}
\renewcommand{\bottomfraction}{0.85}
\renewcommand{\textfraction}{0.15}
\renewcommand{\floatpagefraction}{0.8}
\renewcommand{\textfraction}{0.1}
\setlength{\floatsep}{3pt plus 1pt minus 1pt}
\setlength{\textfloatsep}{3pt plus 1pt minus 1pt}
\setlength{\intextsep}{3pt plus 1pt minus 1pt}
\setlength{\abovecaptionskip}{2pt plus 1pt minus 1pt}

\begin{document}

\title{AER303 Aerospace Laboratory - Supersonic Lab}
%\titlerunning{Add subtitle}

\author{Eric Dai\inst{1} \and Jai Willems\inst{2} \and Mingde Yin\inst{3}}
%\authorrunning{F. Author et al.}

\institute{Division of Engineering Science, University of Toronto, Toronto, Canada \email{eric.dai@mail.utoronto.ca}\\ \and Division of Engineering Science, University of Toronto, Toronto, Canada \email{jai.willems@mail.utoronto.ca}\\ \and Division of Engineering Science, University of Toronto, Toronto, Canada\\ \email{mingde.yin@mail.utoronto.ca}}

\maketitle


%%%%%%%%%%%%
% Abstract %
%%%%%%%%%%%%


\begin{abstract}


\keywords{word_1 \and word_2 \and word_3 \and word_4}
\end{abstract}


%%%%%%%%%%%%%%%%
% Nomenclature %
%%%%%%%%%%%%%%%%


\newpage
\section{Nomenclature}

Refer to Table \ref{tab:nomenclature} for definitions and symbols common to this report.

\begin{table}[h]
    \centering
    \begin{tabular}{p{4.5cm}p{11cm}}
        \toprule
        Symbol/Term & Description \\
        \midrule
        & \\
        & \\
        \bottomrule
    \end{tabular}
    \caption{Commonly used symbols and terms.}
    \label{tab:nomenclature}
\end{table}


%%%%%%%%%%%%%%%%%%%%%%%%%%%%%%%
% Introduction and Background %
%%%%%%%%%%%%%%%%%%%%%%%%%%%%%%%


\newpage
\section{Introduction and Background}\label{sec:introduction_and_background}


%%%%%%%%%%%%%%%%%%%%%%%
% Experimental Set-Up %
%%%%%%%%%%%%%%%%%%%%%%%


\section{Experimental Set-Up}

\noindent
This section details the lab setup and procedure used in the experiment.

\subsection{Apparatus}

\subsection{Procedure}\label{sec:procedure}


%%%%%%%%%%%%%%%%%%%%%%%%%%
% Results and Discussion %
%%%%%%%%%%%%%%%%%%%%%%%%%%


\section{Results and Discussion}


%%%%%%%%%%%%%%%%%%%%
% Sources of Error %
%%%%%%%%%%%%%%%%%%%%


\section{Sources of Error}\label{sec:source_of_error}

\subsection{Quantifiable Sources of Errors}
\begin{table}[H]
\begin{center}
    \begin{tabular}{ll}
        \toprule
        \multicolumn{2}{c}{Quantifiable Error}\\
        \midrule
         & \\
         & \\
        \bottomrule
\end{tabular}
\end{center}
\caption{Quantifiable Sources of Error, Tabulated}
\label{tab:quant_error}
\end{table}


%%%%%%%%%%%%%%
% Conclusion %
%%%%%%%%%%%%%%


\section{Conclusion}


%%%%%%%%%%%%%%%%
% Bibliography %
%%%%%%%%%%%%%%%%


\bibliographystyle{ieeetr}
\bibliography{biblio}


%%%%%%%%%%%%
% Appendix %
%%%%%%%%%%%%


\appendix

\section{Pressure Measurements}

\section{Uncertainty Propagation}

\section{MATLAB and Python Code}

\end{document}